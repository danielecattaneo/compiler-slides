% !TEX root = main.tex

\subsection{The Control Flow Graph}


\begin{frame}{A step back...}
Remember how we described the internal structure of an LLVM-IR module:
\begin{itemize}
\item \cppinline{llvm::Module} is a list of \cppinline{llvm::GlobalValue}s.
	\begin{itemize}
	\item \cppinline{llvm::Function} is a kind of \cppinline{llvm::GlobalValue}.
	\end{itemize}
\item \cppinline{llvm::Function} is a list of \cppinline{llvm::BasicBlock}s.
\item \cppinline{llvm::BasicBlock} is a list of \cppinline{llvm::Instruction}.
\end{itemize}
\bigskip
Functions and basic blocks act like containers:

\begin{itemize}
\item STL-like accessors: \cppinline{front()}, \cppinline{back()},
      \cppinline{size()}, \ldots
\item STL-like iterators: \cppinline{begin()}, \cppinline{end()}
\end{itemize}

\vfill
Each contained element is aware of its container:
\begin{itemize}
\item \cppinline{getParent()}
\end{itemize}
\vfill
Warning for BBs: order of iteration $\neq$ order of execution!
\end{frame}


\begin{frame}{A step back...}
In a \cppinline{llvm::BasicBlock}, the \cppinline{llvm::Instruction}s execute
in the order specified by the list.\\
\begin{itemize}
\item In which order do the \cppinline{llvm::BasicBlock}s execute?
\end{itemize}
\bigskip
The way the basic blocks are executed is implcitly described by
the \alert{branches} in each block.\\
\begin{itemize}
\item These branches describe the \alert{Control Flow Graph} of the function.
\end{itemize}
\end{frame}


\begin{frame}{Control Flow Graph}
LLVM automatically maintains a simple API for operating on the CFG:

\begin{itemize}
\item no need to run passes
\item no need to search the branch instructions in each basic block
\end{itemize}
\bigskip
Every CFG has an \alert{entry} basic block:

\begin{itemize}
\item the \alert{first} executed basic block
\item it is the \alert{root/source} of the graph
\item get it with \cppinline{llvm::Function::getEntryBlock()}
\end{itemize}

\end{frame}


\begin{frame}{Control Flow Graph}{Walking}
At the end of a basic blocks there's always a \alert{terminator} instruction:
\begin{itemize}
\item \llvminline{ret}, \llvminline{br}, \llvminline{switch}, \llvminline{unreachable}, \ldots
\end{itemize}

\bigskip
More than one \alert{exit} block can be present in a function:

\begin{itemize}
\item they are the \alert{leaves/sinks} of the graph
\item their terminator instructions are always \llvminline{ret}s
\begin{enumerate}
\item \cppinline{llvm::BasicBlock::getTerminator()}
\item check the opcode of the terminator
\end{enumerate}
\end{itemize}
\end{frame}


\begin{frame}{Side Note}{Casting Framework}
For performance reasons, a custom casting framework is used:

\begin{itemize}
\item you cannot use \cppinline{static\_cast} and \cppinline{dynamic\_cast} with
      types/classes provided by LLVM
\end{itemize}

\begin{block}{LLVM Casting Functions}
\centering
\medskip
\begin{tabular}{rl}

Static cast of \cppinline{Y*} to \cppinline{X}  &
\cppinline{X *llvm::cast<X>(Y *)}                \\

Dynamic cast of \cppinline{Y*} to \cppinline{X}  &
\cppinline{X *llvm::dyn\_cast<X>(Y *)}            \\

Is \cppinline{Y*} an instance of \cppinline{X}?  &
\cppinline{bool llvm::isa<X>(Y *)} \\

\end{tabular}
\smallskip
\end{block}

Example:

\begin{itemize}
\item is \cppinline{BB} a sink?\\
      \cppinline{llvm::isa<llvm::ReturnInst>(BB.getTerminator())}
\end{itemize}
\end{frame}


\begin{frame}{Control Flow Graph}{Basic Blocks}
Every basic block \cppinline{BB} has one or more\footnote{see include/llvm/IR/CFG.h}:

\begin{description}[predecessors]
\item[predecessors] from \cppinline{pred\_begin(BB)} to
      \cppinline{pred\_end(BB)}
\item[successors] from \cppinline{succ\_begin(BB)} to
      \cppinline{succ\_end(BB)}
\end{description}

\vfill
Other convenience methods available in \cppinline{llvm::BasicBlock}:

\begin{itemize}
\item useful getters
\begin{itemize}
\item \cppinline{BasicBlock *getUniquePredecessor()}
\item \ldots
\end{itemize}
\item moving a basic block
\begin{itemize}
\item      \cppinline{moveBefore(llvm::BasicBlock *)}
\item      \cppinline{moveAfter(llvm::BasicBlock *)}
\end{itemize}
\item split a basic block:
\begin{itemize}
\item      \cppinline{splitBasicBlock(llvm::BasicBlock::iterator)}
\end{itemize}
\item \ldots
\end{itemize}
\end{frame}


\begin{frame}{Control Flow Graph}{Instructions}
The \cppinline{llvm::Instruction} class defines common operations: \\
\medskip
\begin{itemize}
\item getting an operand
\begin{itemize}
\item \cppinline{getOperand(unsigned)}
\end{itemize}
\end{itemize}
\vfill
Subclasses provide specialized accessors: \\
\medskip
\begin{itemize}
\item the \llvminline{load} instruction takes as operand the pointer to the memory to be loaded:
\begin{itemize}
\item      \cppinline{llvm::LoadInst::getPointerOperand()}
\end{itemize}
\end{itemize}
\end{frame}


\begin{frame}{Instructions}{Creating New Instructions}
Instructions are created using:

\begin{itemize}
\item constructors
\begin{itemize}
\item \cppinline{llvm::LoadInst::LoadInst(...)}
\end{itemize}
\item factory methods
\begin{itemize}
\item \cppinline{llvm::GetElementPtrInst::Create(...)}
\end{itemize}
\item the helper class \cppinline{llvm::IRBuilder}
\begin{itemize}
\item \cppinline{llvm::IRBuilder<> builder(insPoint);}\\
\cppinline{builder.CreateAdd(...);}
\end{itemize}
\end{itemize}
\vfill
\alert{Interface is not homogeneous!}\\
Some instructions support all methods, others support only one.
\end{frame}


\begin{frame}{Instructions}{Inserting New Instructions}
\vfill
Instructions can be inserted:
\vfill
\begin{itemize}
\item automatically by \cppinline{IRBuilder}
\begin{itemize}
\item insertion point is given at \cppinline{IRBuilder} instantiation
\end{itemize}
\bigskip
\item manually by appending to a basic block
\item manually by inserting after/before another instruction
\end{itemize}
\vfill
\end{frame}


\begin{frame}{From Control Flow to Data Flow}{Definitions and Uses}
In LLVM, the data flow generated by the various instructions is represented by a simple hierarchy:

\begin{description}[valueMMM]
\item[value] something that can be used: \cppinline{llvm::Value}
\item[user] something that can use: \cppinline{llvm::User}
\item[use] the link between the \alert{value} and the \alert{user}: \cppinline{llvm::Use}
\end{description}
\medskip
A value is a \alert{definition}:

\begin{itemize}
\item Visiting where a definition is used:
\begin{itemize}
\item \cppinline{llvm::Value::use\_begin()}, \cppinline{llvm::Value::use\_end()}
\end{itemize}
\end{itemize}
\medskip
An user accesses \alert{definitions}:

\begin{itemize}
\item Visiting the definitions that are used:
\begin{itemize}
\item \cppinline{llvm::User::op\_begin()}, \cppinline{llvm::User::op\_end()}
\end{itemize}
\end{itemize}
\medskip

\end{frame}


\begin{frame}{From Control Flow to Data Flow}{Instructions are Values}
\vfill
\begin{itemize}
\item \cppinline{llvm::Value} inherits from \cppinline{llvm::User}
\item \cppinline{llvm::Instruction} inherits from \cppinline{llvm::Value}
\begin{itemize}
\normalsize
\item[$\Rightarrow$] The value produced by the instruction is\\the \alert{instruction itself}!
\end{itemize}
\end{itemize}

\begin{block}{Example}
\begin{center}
\llvminline{\%6 = load i32, i32* \%1, align 4}\\
\medskip
The \llvminline{load} is described by an instance of \cppinline{llvm::Instruction}. \\
That instance also represents the \llvminline{\%6} variable. \\
\end{center}
\end{block}

\begin{center}
Not all instances of \cppinline{llvm::Value} are also \cppinline{llvm::Instruction}s!\\
\smallskip
{\small i.e. function arguments}

\end{center}
\vfill
\end{frame}


\begin{frame}{From Control Flow to Data Flow}{Value Typing}
Every \cppinline{llvm::Value} is typed:

\begin{itemize}
\item use \cppinline{llvm::Value::getType()} to get the type
\end{itemize}

\vfill
Since every instruction is a value:

\begin{itemize}
\item instructions are typed
\end{itemize}

\vfill
\begin{block}{Example}
\begin{center}
\llvminline{\%6 = load i32, i32* \%1, align 4}\\
\medskip
The type of the \llvminline{\%6} variable is the type of the return value of the \llvminline{load} instruction, \llvminline{i32}\\
\end{center}
\end{block}
\end{frame}


%\begin{frame}{From Control Flow to Data Flow}{Instructions are like functions}
%\vfill
%\vfill
%\begin{columns}[t]
%\column{.45\textwidth}
%Functions:
%
%\begin{itemize}
%\item used by call sites
%\item uses formal parameters
%\end{itemize}
%
%\column{.45\textwidth}
%Instructions:
%
%\begin{itemize}
%\item define an SSA value
%\item uses operands
%\end{itemize}
%\end{columns}
%
%\vfill
%\vfill
%\end{frame}


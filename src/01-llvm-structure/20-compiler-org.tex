% !TEX root = main.tex

\section{Compiler organization}


\begin{frame}{Compiler pipeline}
\begin{center}
Typically a compiler is a \alert{pipeline}.\\
\bigskip
Advantages of the pipeline model:\\
\medskip
\begin{varwidth}{0.8\textwidth}
\begin{itemize}
\item \alert{simplicity} -- read something, produce something
\item \alert{locality} -- no superfluous state
\end{itemize}
\end{varwidth}
\\
\bigskip
Complexity lies on \alert{chaining} together stages.
\end{center}
\end{frame}


\begin{frame}{Frontends and Drivers}{C Language Family Front-end}
The external interface to LLVM is provided by the \alert{compiler driver}.\\
\bigskip
The \alert{compiler driver} is the program that:
\begin{itemize}
\item Provides the interface to the user
\item Performs setup of the frontend and LLVM itself. 
\end{itemize}
\bigskip
The driver invokes the \alert{first stage of the pipeline}, the \alert{frontend}.
\begin{example}
\alert{Clang}\cite{LOCAL:www/clang} is the frontend for the C language family.\\
The driver of \emph{Clang} is the \texttt{clang} executable (compatible with GCC).\\
\smallskip % why does \cite screw up the f*ing height of my block!?
\end{example}
\end{frame}


\begin{frame}{Compiler pipeline}{Internal pipelines}
\begin{center}
High-level pipeline structure of a compiler:\\
\begin{figure}
\centering
\input{img/compiler-structure.tex}
\end{figure}
\medskip
There are three main components:

\begin{description}
\item[Front-end] \alert{translates} a source file into the intermediate representation
\item[Middle-end] \alert{analyzes} the intermediate representation, \alert{optimizes}
                  it
\item[Back-end] \alert{generates} target machine assembly from the intermediate
                representation
\end{description}
\end{center}
\end{frame}


\begin{frame}{The LLVM compiler pipeline}
\begin{center}
We will focus on the \emph{middle-end}.\\
{\small Same concepts are valid also for \{front,back\}-end.}\\
\end{center}
\bigskip
\begin{itemize}
\item The \emph{front-end} is completely language-specific
	\begin{itemize}
	\item There are no special facilities in LLVM for parsing and AST generation
	\item This is slowly changing: \alert{MLIR}
	\end{itemize}
\item The \emph{back-end} uses its own machine-specific IR
	\begin{itemize}
	\item As opposed to the frontend, in LLVM there are generalized facilities for implementing a backend
	\item Complex topic, not enough time...
	\item Most optimizations happen in the middle-end anyway
	\end{itemize}
\end{itemize}
\end{frame}


\begin{frame}{The LLVM compiler pipeline}
\begin{center}
The lingua franca of LLVM is its \alert{Intermediate Representation} called\\
\medskip
{\large LLVM-IR}\\
\end{center}
\bigskip
The LLVM-IR is:
\begin{itemize}
\item \alert{Produced} by the \alert{front-end}
\item \alert{Transformed and optimized} by the \alert{middle-end}
\item \alert{Consumed} by the \alert{back-end}
\end{itemize}
\bigskip
Understanding LLVM-IR is the key to hacking within LLVM.
\end{frame}


\begin{frame}{Remember...}
\begin{center}
LLVM is a \alert{compiler construction framework}\\
It operates on the \alert{LLVM-IR} language.\\
\bigskip
$\Downarrow$\\
\bigskip
Using LLVM \emph{by itself} does not make much sense!\\
Writing LLVM-IR by hand is unfeasible.
\end{center}
\end{frame}


\begin{frame}{The Middle End}
\centering
The middle-end is not a monolithic block,\\but it is a pipeline in and of itself.\\
\raggedright
\bigskip
\begin{description}[Pass Manager]
\item[LLVM-IR] the \alert{language} used in the middle-end
\item[Pass] a \alert{pipeline stage}\\
a Pass may have \alert{dependencies} on other Passes.
\item[Pass Manager] component that \alert{schedules} passes according to their \alert{dependencies} and \alert{executes} them\\
\emph{(builds the pipeline)}
\end{description}
\bigskip\centering
Our focus: \alert{writing a pass}
\end{frame}


\begin{frame}{First insights}
A compiler is \alert{complex}:

\begin{itemize}
\item passes are the \alert{elementary unit of work}
\item Pass Manager must be \alert{advised} about pass chaining
\item pipeline shapes are \alert{not fixed} -- they can change from one compiler
      execution to another\\
      {\small e.g. optimized/not optimized builds, compiler options, ...}
\end{itemize}
\end{frame}


\begin{frame}{A word of warning!}
\begin{center}
{\large 
Compilers must be \alert{conservative}:\\
\smallskip
$\Downarrow$\\
\smallskip
All passes \alert{must preserve the program semantics}\\
\smallskip
$\Downarrow$\\
\smallskip
Compiler passes must be designed \alert{very carefully}!\\
}
\end{center}
\end{frame}

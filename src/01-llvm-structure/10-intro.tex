% !TEX root = main.tex

\section{Introduction}


\begin{frame}{Compilers and compilers}
\begin{center}
You might have already have had experience working in a \alert{toy compiler}... 

\begin{columns}[t]

\column{.40\textwidth}
\begin{block}{Toy Compiler}
\begin{itemize}
\item small codebase
\item easy to modify
\item limited capabilities
\end{itemize}
\end{block}

\column{.50\textwidth}
\begin{block}{Production-Quality Compiler}
\begin{itemize}
\item huge codebase
\item hard to modify
\item produces high-quality code
\end{itemize}
\end{block}

\end{columns}
\bigskip
\emph{Initially}, working with a production-quality compiler might seem \alert{hard}...\\
\medskip
...however it provides a huge set of tools that toy compilers \alert{miss}!
\end{center}
\end{frame}


\begin{frame}{Why LLVM}
\begin{center}
Why is this course focused on LLVM?\\
\end{center}
\begin{itemize}
\item Key technology in the \alert{industry}
	\begin{itemize}
	\item AMD, Apple, Google, Intel, NVIDIA...
	\end{itemize}
\item Biggest platform for \alert{research} about compilers
\item \alert{Modular} and \alert{hackable}
\end{itemize}
\bigskip
\begin{center}
Initially started as a small research project at Urbana-Champaign.\\
\medskip
Now it has grown to a huge size...
\end{center}
\end{frame}


\begin{frame}{GCC vs LLVM}
\begin{center}
\textbf{LLVM}~\cite{LOCAL:www/llvm} is Open Source\\
\medskip
If you are familiar with Linux you might have used \textbf{GCC}~\cite{LOCAL:www/gcc}...\\
\bigskip
GCC is older than LLVM\\
\bigskip
\bigskip
\vbox{
\begin{varwidth}{6cm}
\begin{itemize}
\item[$\Rightarrow$] GCC produces better code
\item[$\Rightarrow$] LLVM is generally faster
\item[$\Rightarrow$] LLVM is more modular and \emph{clean}
\end{itemize}
\end{varwidth}
}
\end{center}
\end{frame}


% !TEX root = main.tex

\section{Introduction}


\begin{frame}{TODO}
\begin{itemize}
\item Overview of things made possible by DWARF
\item Predecessors of DWARF
\item Platforms using DWARF
\item Presentation of topics
\end{itemize}
\end{frame}


\begin{frame}{A normal day in a programmer's life}
\begin{itemize}
\item Imagine you are debugging a program and you can't find a bug...
\item You decide to single-step it in a \alert{debugger}
\item You get:
	\begin{itemize}
	\item Evaluation of variables and expressions
	\item Examination of stack frames
	\item Current instruction pointer shown in the source code
	\end{itemize}
\pause
\item \textbf{How does it all work?}
\end{itemize}
\end{frame}


\begin{frame}{Inside the debugger}
Debuggers consist of 2 parts:
\begin{enumerate}
\item A \alert{monitor} component
	\begin{itemize}
	\normalsize
	\item Implements controlling the program
	\item Start, Stop, Restart
	\item \alert{Breakpoints}, Single-Stepping...
	\item Platform-dependent and operating-system dependent
	\end{itemize}
\item An \alert{analysis} component
	\begin{itemize}
	\normalsize
	\item Collects information about where things are in the program
	\item Memory layout, data types...
	\item Memory address for each source code line
	\end{itemize}
\end{enumerate}
\end{frame}


\begin{frame}{Inside the debugger}
\begin{center}
The analysis components needs to get the information from somewhere...\\
\bigskip
And it cannot simply reimplement the compiler inside itself!\\
\medskip
The user may have compiled their code however they liked, reversing the transformations
done by the compiler is next to impossible.\\
\bigskip
\large
We need a \alert{data structure} that tells us where things are inside the compiled
executable!
\end{center}
\end{frame}


\begin{frame}{Inside the debugger}
This data structure should be:
\begin{itemize}
\item Extensible
\item Language-independent
\item Compact
\item Self-contained within the executable, or optionally delivered separately
\end{itemize}
\end{frame}


\begin{frame}{Inside the debugger}
\begin{center}
We are talking about the \alert{debugging information}\\
\smallskip
Basically it's the data that the compiler adds when you specify the \texttt{-g} flag\\
\bigskip
\large
The specification that describes this data structure is called \alert{DWARF}.\\
\footnotesize
\smallskip
``Debugging With Attributed Record Formats''\\
Yeaaah we all believe that...
\end{center}
\end{frame}


\begin{frame}{DWARF}
\begin{itemize}
\item Originally developed at Bell Labs together with the ELF object file format in 1988
\item Standardized in 1992, but takes its current form in 1993 (DWARF 2).
	\begin{itemize}
	\item Adoption is not istantaneous, DWARF coexists with a simpler format named ``stabs''
	\end{itemize}
\item Becomes an open standard roughly at the turn of the century, adoption continues to grow
\item Nowadays DWARF is the most used debugging format
	\begin{itemize}
	\item All Unixes use DWARF as the standard debugging information format
	\item Open-source compiler toolchains (GDB, LLVM) use DWARF in embedded development
	\item DWARF is stored in standard sections in the executable, and it is loaded 
		alongside the code: even OSes or binary formats
		not explicitly designed for it can support its usage
	\end{itemize}
\end{itemize}
\end{frame}


\begin{frame}{Other debugging formats}
\begin{itemize}
\item Windows uses the CodeView debugging format. It is usually stored in separate files (\texttt{.pdb} extension)
\item ``stabs'' is still available for use in most Unixes but nobody uses it anymore
\item Other debugging formats exist but are either outdated or nobody uses them
\end{itemize}
\end{frame}


\begin{frame}{The structure of DWARF}
\begin{center}
At the highest level of abstraction, DWARF consists of a \alert{tree} structure.\\
\medskip
Each \alert{node} in the tree corresponds to an element of the program.\\
{\small General idea is similar to a compiler's AST.}\\
\smallskip
These nodes are named \alert{DIEs} (Debugging Information Entries).\\
\end{center}

\bigskip
Every DIE has:
\begin{enumerate}
\item a \alert{type} (named \alert{tag})
\item a set of \alert{key-value} pairs containing the debug information itself
\end{enumerate}
\end{frame}


\begin{frame}{The structure of DWARF}
\begin{center}
You can examine the DIEs in an object file by using\\
\texttt{\textbf{dwarfdump}}\\
\bigskip
In the following discussion we will often see output from \texttt{llvm-dwarfdump}\\
\medskip
{\footnotesize Same thing as normal GNU \texttt{dwarfdump}, but formats output in a clearer way}\\
\end{center}
\end{frame}


\begin{frame}{Up next...}
\begin{enumerate}
\item We will see \alert{how DWARF models the memory of the program} using DIEs
	\begin{itemize}
	\item Where data is stored
	\item Where the code is stored
	\item Other miscellaneous information
	\end{itemize}
\item Then we will briefly go over how \alert{DWARF serializes this information inside the object files}
\end{enumerate}
\bigskip
\centering Let's get started!\\

\end{frame}



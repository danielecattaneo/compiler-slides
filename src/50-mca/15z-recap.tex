% !TEX root = main.tex

\subsection{Recap}


\begin{frame}{Recap}
\begin{itemize}
\item LLVM-MCA analyzes the CPU performance of your machine code quite accurately
\item \alert{LLVM-MCA does not take into account the overhead of the memory hierarchy}
\item Analyze the views provided by LLVM-MCA carefully! Do not get tricked by irrelevant metrics.
\item Improve the code one bottleneck at a time
\item Perform large improvements first
\item When you find out that you are overfitting on your target machine, consider stopping
\medskip
\item {\footnotesize And it should go without saying, but...}\\
	{\normalsize\alert{Start with a good algorithm before you even consider using LLVM-MCA!}}
\end{itemize}
\end{frame}


\begin{frame}{Final Remarks}
\begin{itemize}
\item Most of the optimizations we made manually are applied automatically 
by LLVM when you use the \texttt{-ffast-math} flag
	\begin{itemize}
	\item This does not mean the techniques we saw are useless!
	\item It's just that our example was really simple
	\end{itemize}
\medskip
\item A further optimization that can be made is \alert{vectorization}
	\begin{itemize}
	\item Try it at home!
	\item Example code (alongside with all other examples shown here!):\\
				\url{https://github.com/danielecattaneo/LLVM-intro/tree/master/src/50-mca/example}
	\item We can lower the execution time to \SI{0.044824}{\second}
	\end{itemize}
\medskip
\item Not all CPUs are supported by LLVM-MCA!
	\begin{itemize}
	\item Check if yours is supported or not using \texttt{llc --version}!
	\end{itemize}
\end{itemize}
\end{frame}



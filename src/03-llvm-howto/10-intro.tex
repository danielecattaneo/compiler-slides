% !TEX root = main.tex

\section{Introduction}


\begin{frame}{Understanding LLVM}
	\begin{center}
	\huge{
		LLVM is \textbf{not} a compiler.\\
		\pause
		\vfill
		LLVM is a\\\textbf{collection of components}\\
		which is \textbf{useful}\\to build a compiler.
	}
	\end{center}
\end{frame}


\begin{frame}{Getting LLVM}
All work on LLVM goes to the \alert{monorepo} on GitHub\\
\bigskip
%\begin{itemize}
%	\item git monorepo
		\begin{itemize}
			\item It contains LLVM + major subprojects handled by the LLVM project
			\item \texttt{git clone -b release/18.x --single-branch git@github.com:llvm/llvm-project.git}
		\end{itemize}
%\end{itemize}
%\bigskip
%{\small A few years ago they used a private SVN repository, the switch to GitHub is recent}
\end{frame}


\begin{frame}{What LLVM is made of}
\begin{itemize}
	\item C++ libraries
		\begin{itemize}
			\item \texttt{llvm/include/llvm/...}
			\item \texttt{llvm/lib/...}
		\end{itemize}
		\vfill
	\item small application (tools)
		\begin{itemize}
			\item \texttt{llvm/tools/...}
			\item \texttt{llvm/utils/...}
		\end{itemize}
\end{itemize}
\vfill
Binaries installed under \texttt{bin/...}
\end{frame}


\begin{frame}{Commands}
	\begin{description}[llvm-dwarfdump]
		\item[llvm-as] LLVM assembler
		\item[llvm-dis] LLVM disassembler
		\item[opt] LLVM optimizer
		\item[llc] LLVM static compiler
		\item[lli] directly execute programs from LLVM bitcode
		\item[llvm-link] LLVM bitcode linker
		\item[llvm-mca] LLVM machine code analyzer
		\item[llvm-nm] list LLVM bitcode and object file's symbol table
		\item[llvm-stress] generate random .ll files
		\item[llvm-config] prints out install configuration parameters
		\item[llvm-dwarfdump] print contents of DWARF sections
	\end{description}
	\vfill
	For a complete reference, see the LLVM command guide\footnote{\url{http://llvm.org/docs/CommandGuide/index.html}}
\end{frame}



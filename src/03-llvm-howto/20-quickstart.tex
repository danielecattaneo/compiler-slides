% !TEX root = main.tex

\section{LLVM framework quick start}


\begin{frame}{Simulating a LLVM driver manually}
%	\noindent\hspace{-1.2cm}\includegraphics[width=13cm]{img/toolchain}
\begin{centering}
\begin{tikzpicture}[yscale=0.55,xscale=0.6]
\node[anchor=west] (step0) at (0,0) {\texttt{.c} source};
\node[anchor=west] (step1) at (1,-1) {\texttt{\textbf{clang} -emit-llvm}};
\node[anchor=west] (step2) at (2,-2) {\texttt{.bc}/\texttt{.ll}};
\node[anchor=west] (step3) at (3,-3) {\texttt{\textbf{llvm-link}}};
\node[anchor=west] (step4) at (4,-4) {\texttt{.bc}/\texttt{.ll}};
\node[anchor=west] (step5) at (5,-5) {\texttt{\textbf{opt}}};
\node[anchor=west] (step6) at (6,-6) {\texttt{.bc}/\texttt{.ll}};
\node[anchor=west] (step7) at (7,-7) {\texttt{\textbf{llc}}};
\node[anchor=west] (step8) at (8,-8) {\texttt{.s}};
\node[anchor=west] (step9) at (9,-9) {\textbf{\texttt{llvm-mc}/\texttt{as}}};
\node[anchor=west] (step10) at (10,-10) {\texttt{.o}};
\node[anchor=west] (step11) at (11,-11) {\textbf{\texttt{lld}/\texttt{ld}}};
\node[anchor=west] (step12) at (12,-12) {executable};
\foreach \i in {0,...,11}{
  \draw[->] ($(\i+0.5,-\i-0.5)$) |- ($(\i+1,-\i-1)$);
}
\node[anchor=south west, above right=0.2ex and 1.5em of step3.north east] (note0) {\texttt{otherModule.ll/.bc}};
\draw[->] ($(note0.south west)+(0.5em,0.5ex)$) -- (step3.east);
\node[anchor=south west, above right=0.2ex and 1em of step11.north east] (note1) {\texttt{libFoo.a/.so}};
\draw[->] ($(note1.south west)+(0.5em,0.5ex)$) -- (step11.east);
\node[anchor=south west,draw,dashed,very thick,rounded corners=5pt,align=center] at (0,-12) 
  {\texttt{.ll} $\rightarrow$ \texttt{\textbf{llvm-as}} $\rightarrow$ \texttt{.bc}\\
  \texttt{.bc} $\rightarrow$ \texttt{\textbf{llvm-dis}} $\rightarrow$ \texttt{.ll}};
\end{tikzpicture}\\
\end{centering}
\end{frame}


\begin{frame}{Writing a LLVM pass}
	There are a lot of tutorials available:
	\vfill
	\begin{itemize}
		\item Official developer guide\\ \href{https://llvm.org/docs/WritingAnLLVMNewPMPass.html}{\url{https://llvm.org/docs/WritingAnLLVMNewPMPass.html}}
		\vfill
		\item Out-of-source pass (outdated!)\\ \href{https://github.com/quarkslab/llvm-dev-meeting-tutorial-2015}{\url{github.com/quarkslab/llvm-dev-meeting-tutorial-2015}}
	\end{itemize}
	\vfill
	We will follow the first one, with a few adjustments.
\end{frame}


\begin{frame}{Building LLVM}
\begin{center}
To test your pass you need a \alert{Debug+Assertions} build of LLVM.\\
\bigskip
This build needs to be \alert{kept separated} from normal Release builds (it's very slow!)\\
\bigskip
The best way to get such a LLVM build is to \alert{make it yourself}!
\end{center}
\end{frame}


\begin{frame}{Building LLVM}
\begin{itemize}
\item Detailed instructions: \url{https://llvm.org/docs/GettingStarted.html}
\end{itemize}
\bigskip
\begin{description}
\item[Problem 1] With the \alert{default options}, a finished build takes\\\alert{25 GB of disk space}
\item[Problem 2] A standard build with the GNU toolchain uses \alert{a lot of RAM} ($\approx$16 GB or more with a modern 4 core CPU!) especially when linking
\end{description}
\bigskip
We need to customize the build process a bit...
\end{frame}


\begin{frame}{Building LLVM}
\begin{itemize}
\item The build flags I recommend:\\
\begin{itemize}
\item[]\texttt{\small-GNinja \\
-DCMAKE\_BUILD\_TYPE=Debug\\
-DLLVM\_ENABLE\_PROJECTS='clang;compiler-rt' \\
-DLLVM\_INSTALL\_UTILS=ON \\
\textbf{-DLLVM\_BUILD\_LLVM\_DYLIB=ON \\
-DLLVM\_LINK\_LLVM\_DYLIB=ON} \\
-DLLVM\_OPTIMIZED\_TABLEGEN=ON \\
-DLLVM\_INCLUDE\_EXAMPLES=OFF \\
-DCMAKE\_INSTALL\_PREFIX=/opt/llvm-18-d \\
%\textbf{-DLLVM\_USE\_LINKER=lld \\
%-DCMAKE\_C\_COMPILER=clang-18 \\
%-DCMAKE\_CXX\_COMPILER=clang++-18} \\
\textbf{-DLLVM\_PARALLEL\_LINK\_JOBS=1}\\
}
\end{itemize}
%\item Building with LLVM itself solves the RAM usage problem!
%	\begin{itemize}
%	\item Not required on macOS or *BSD: they already ship LLVM as default
%	\end{itemize}
\item Using \alert{shared libraries} drops the disk usage to \alert{10 GB}.\\
{\footnotesize The build products alone will still take 20 GB of disk space...}
\item \texttt{-DLLVM\_PARALLEL\_LINK\_JOBS=1} mitigates GNU \texttt{ld} RAM usage\\
{\footnotesize Or you could use \texttt{clang}+\texttt{lld}...}
	\begin{itemize}
	\item Not required on macOS or *BSD: they don't use GNU \texttt{ld}
	\end{itemize}
\end{itemize}
\end{frame}


\begin{frame}{Last notes on building}
You can add other projects to the LLVM build by modifying the value of the \texttt{LLVM\_ENABLE\_PROJECTS} flag\\
\bigskip
Good practice: \alert{always include clang}
\begin{itemize}
	\item You can easily see on a production quality compiler the impact of changes you made on your local copy of LLVM
\end{itemize}
\bigskip
To install cutting-edge release LLVM if your Linux distribution does not provide it:\\
\begin{itemize}
	\item \url{https://apt.llvm.org}
\end{itemize}
\end{frame}


\begin{frame}{Testing}
LLVM has an internal testing infrastructure\footnote{\url{http://llvm.org/docs/TestingGuide.html}}.
Please use it.
\\
\begin{description}
	\item[llvm-lit] LLVM Integrated Tester
\end{description}
\begin{enumerate}
	\item Forge a proper LLVM-IR input file (.ll) for your test case
	\item Instrument it with \texttt{lit} script comments
	\item Run \texttt{lit} on your test
		\begin{itemize}
			\item \texttt{llvm-lit ~/llvm/test/myTests/singleTest.ll}\\ run a single test
			\item \texttt{llvm-lit ~/llvm/test/myTests}\\ run the test suite (folder)
		\end{itemize}
	\item Run \texttt{lit} on the LLVM test suite (regression testing)
\end{enumerate}
\vfill
To submit a bug report to LLVM developers you will be asked to write a \texttt{lit} test case that highlights the bug.
\end{frame}

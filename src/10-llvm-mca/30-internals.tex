% !TEX root = main.tex

\section{Internals of LLVM-MCA}


\let\OldFootnoterule\footnoterule
\renewcommand*\footnoterule{}

\begin{frame}{Quick look at how LLVM-MCA works}
There are two main parts to LLVM-MCA:\\
\smallskip
\begin{itemize}
\item The simulation code
	\begin{itemize}
	\item Frontend: \texttt{llvm/tools/llvm-mca}
	\item Backend: \texttt{llvm/lib/MCA}
	\item All simulated machines use \emph{the same model}, just with different parameters
	\item You can include \texttt{llvm/MCA/Context.h} and link with LLVM to use LLVM-MCA as a library!
	\end{itemize}
\smallskip
\item The machine models
	\begin{itemize}
	\item \texttt{llvm/lib/Target/XXX/XXXScheduleYYY.td}\\(naming convention may vary)
		\begin{description}[XXXX]
		\item[XXX] Platform (X86, ARM, PowerPC...)
		\item[YYY] Microarchitecture (Skylake, [Cortex] A57, G5)
		\end{description}
	\item Determines instruction latency, reservation stations available, issue width...
	\end{itemize}
\end{itemize}

\footnotetext[1]{\tiny For those who got their BSc at PoliMi: \alert{don't git-blame llvm-mca}}
\end{frame}

\let\footnoterule\OldFootnoterule


\begin{frame}{What's a .td file?}
LLVM scheduling models are defined in \alert{.td} files\\
\medskip
\begin{itemize}
\item \texttt{.td} stands for \alert{Tablegen Definition}
\item It's a \alert{standard syntax} of the LLVM project for \alert{domain-specific languages}
\item These DSLs are compiled by a tool named \texttt{llvm-tblgen}
\item The usual compilation output for these DSLs are \alert{data structures} (hence, \alert{tables}) containing parameters used by LLVM at runtime
\medskip
\item More info: \url{https://llvm.org/docs/TableGen/index.html}
\end{itemize}
\end{frame}


\begin{frame}{Machine Scheduler Definitions}
\begin{itemize}
\item Root class of the scheduling model: \cppinline{llvm::MCSchedModel}
	\begin{itemize}
	\item See \texttt{llvm/MC/MCSchedule.h}
	\end{itemize}
\item The actual information is contained in \cppinline{llvm::SchedMachineModel}
	subclasses auto-generated from the \texttt{.td} files

\medskip
\item More info in slides from previous LLVM developer meetings: \\\smallskip
\url{https://llvm.org/devmtg/2014-10/Slides/Estes-MISchedulerTutorial.pdf}\\\smallskip
\url{https://llvm.org/devmtg/2016-09/slides/Absar-SchedulingInOrder.pdf}
\end{itemize}
\end{frame}


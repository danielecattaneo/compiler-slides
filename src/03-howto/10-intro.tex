% !TEX root = main.tex

\section{Introduction}


\begin{frame}{Understanding LLVM}
	\begin{center}
	\huge{
		LLVM is \textbf{not} a compiler.\\
		\pause
		\vfill
		LLVM is a\\\textbf{collection of components}\\
		which is \textbf{useful}\\to build a compiler.
	}
	\end{center}
\end{frame}


\begin{frame}{Getting LLVM}
\begin{itemize}
	\item ``old'' git mirrors
		\begin{itemize}
			\item only llvm repo (subprojects in separated repos, can be added later)
			\item \texttt{git clone -b release\_90 --single-branch git@github.com:llvm-mirror/llvm.git}
		\end{itemize}
	\vfill
	\item ``new'' git monorepo
		\begin{itemize}
			\item all in one repo (llvm + major subprojects)
			\item \texttt{git clone -b release/9.x --single-branch git@github.com:llvm/llvm-project.git}
		\end{itemize}
\end{itemize}
\end{frame}


\begin{frame}{What LLVM is made of}
\begin{itemize}
	\item C++ libraries
		\begin{itemize}
			\item \texttt{src/include/llvm/...}
			\item \texttt{src/lib/...}
		\end{itemize}
		\vfill
	\item small application (tools)
		\begin{itemize}
			\item \texttt{src/tools/...}
			\item \texttt{src/utils/...}
		\end{itemize}
\end{itemize}
\vfill
You can find binaries of them in the installation directory under \texttt{root/bin/...}
\end{frame}


\begin{frame}{clang}
\begin{itemize}
	\item \texttt{clang} is a compiler based on LLVM
	\item It compiles all major C-like languages
	\vfill
	\item It is part of the git monorepo
	\item It can be added as a tool in the LLVM framework but must be manually cloned in the tool directory
	\begin{enumerate}
		\item \texttt{cd src/tools}
		\item \texttt{git clone http://llvm.org/git/clang} (git mirror version)
	\end{enumerate}
	\vfill
	\item You can easily see on a production quality compiler the impact of changes you made on your local copy of LLVM
\end{itemize}
\end{frame}

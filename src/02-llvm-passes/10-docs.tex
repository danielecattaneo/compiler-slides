% !TEX root = main.tex

\section{Documentation}
\begin{frame}[t]{LLVM official documentation}
  \begin{center}
    \begin{Huge}
      \vfill
      \href{http://llvm.org/docs}{llvm.org/docs}
      \vfill
    \end{Huge}
  \end{center}
\end{frame}


\begin{frame}[t]{A lot of documentation...}
\vfill
  \href{http://llvm.org/docs}{llvm.org/docs} links to:
  \begin{itemize}
    \item \texttt{\ 4} references about \textit{Design \& Overview}
    \item \texttt{\ 7} references about \textit{Getting Started / Tutorials}
    \item \texttt{53} references about \textit{User Guides}
    \item \texttt{41} references about \textit{Reference Documentation}
    \item \texttt{\ 5} references about \textit{Development Process}
    \item \texttt{\ 4} Forums and Mailing Lists
    \item ...
  \end{itemize}
  \bigskip
  Most of the above references are \alert{outdated}!\\
  \medskip
  You probably need documentation \emph{about the documentation}.
\vfill
\end{frame}


\begin{frame}[t]{Essential documentation}
\vfill
\raggedright
\textbf{Intro to LLVM}~\cite{LOCAL:www/llvmIntro}\\
\begin{itemize}
\item Quick and clear introduction. Details are a bit outdated.
\end{itemize}
\textbf{Writing an LLVM pass}~\cite{LOCAL:www/llvmWritingAPass}\\
\begin{itemize}
\item Explains step by step how to implement a Pass
      for those who never did anything like that.\\
      {\footnotesize (We will see this tutorial later in the course)}
\end{itemize}
\textbf{Doxygen}~\cite{LOCAL:www/llvmDoxygen}\\
\begin{itemize}
\item \textit{The best code documentation is the code itself.}
       Sometimes the generated doxygen documentation is enough.
       Available for development and stable branches.
\end{itemize}
\textbf{LLVM Discourse}~\cite{LOCAL:www/llvmDiscourse}\\
\begin{itemize}
\item Discourse forum. Last resource: ask other developers.
\end{itemize}
\vfill
\end{frame}


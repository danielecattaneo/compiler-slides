% !TEX root = main.tex

\section{LLVM Analyses}


\begin{frame}{Checking Input Properties}
Analyses basically allow to:
\begin{itemize}
\item \alert{derive} information and properties of the input
\item \alert{verify} properties of input
\end{itemize}

\vfill
Keeping analyzed information updated is expensive:
\begin{itemize}
\item tuned algorithms update information when an optimization
      invalidates it
\item incrementally updating analyses are cheaper than recomputing them
\end{itemize}

\vfill
As an \alert{optimization}, many LLVM analysis supports incremental updates.
\end{frame}


\begin{frame}{Useful Analyses}
\begin{center}
We will see the following passes:\\
\bigskip
\begin{tabular}{ccc}
\toprule

\multicolumn{1}{c}{\textbf{Pass}}        &
\multicolumn{1}{c}{\textbf{Name}}      &
\multicolumn{1}{c}{\textbf{Transitive}} \\

\midrule

Dominator tree    &
\texttt{domtree}  &
No               \\

Post-dominator tree   &
\texttt{postdomtree}  &
No                   \\

Loop information  &
\texttt{loops}    &
Yes              \\

Scalar evolution           &
\texttt{scalar-evolution}  &
Yes                       \\

Alias analysis    &
---  &
Yes              \\

Memory SSA  &
\texttt{memoryssa}    &
Yes               \\

\bottomrule
\end{tabular}
\end{center}
\end{frame}


%With new pass manager this topic doesn't fit in a single slide
%People need to read: https://llvm.org/docs/NewPassManager.html
%
%\begin{frame}{Requesting an Analysis}
%\begin{center}
%Your pass needs to tell the pass manager which analyses it needs!
%
%\vfill
%Transitive analyses:\\
%\cppinline{llvm::AnalysisUsage::addRequiredTransitive<T>()}\\
%\medskip
%Non-transitive analyses: \\
%\cppinline{llvm::AnalysisUsage::addRequired<T>()}\\
%
%\vfill
%For \alert{chained analyses}\footnote{Analyses that use the result of another analysis}, the \texttt{addRequiredTransitive} method
%should be used instead of the \texttt{addRequired} method.\\
%\medskip
%{\small
%This informs the PassManager that the transitively required pass
%should be alive as long as the requiring pass is.}
%\end{center}
%\end{frame}


\subsection{Dominance Trees}


\begin{frame}{Dominance Trees}
Dominance trees answer to control-related queries:\\

\vfill
\begin{columns}[onlytextwidth]
\column{.475\textwidth}
\begin{centering}
is \texttt{A} executed \alert{before} \texttt{B}?\\
$\Downarrow$\\
\cppinline{llvm::DominatorTree}\\
\end{centering}

\column{.525\textwidth}
\begin{centering}
is \texttt{A} executed \alert{after} \texttt{B}?\\
$\Downarrow$\\
\cppinline{llvm::PostDominatorTree}\\
\end{centering}
\end{columns}

\vfill
The interfaces of these two trees is mostly the same:

\begin{itemize}
\item \cppinline{bool dominates(A, B)}
\item \cppinline{bool properlyDominates(A, B)}
\end{itemize}

\cppinline{A} and \cppinline{B} are either
\cppinline{llvm::BasicBlock}s or
\cppinline{llvm::Instruction}s

\vfill
By using \texttt{opt}, it is possible to show the trees:

\begin{itemize}
%\item \aligntext{\texttt{-view-postdom},}{\texttt{-view-dom},} \texttt{-dot-dom}
%\item \aligntext{\texttt{-view-postdom},}{\texttt{-view-postdom},} \texttt{-dot-postdom}
\item \texttt{print<domtree>}, \texttt{print<postdomtree>}
\end{itemize}
\end{frame}


\subsection{Loop Information}


\begin{frame}{Loop Information}
Loop information is represented using two classes:

\begin{description}[\texttt{llvm::LoopInfo}]
\item[\texttt{llvm::LoopInfo}]
			The result of \cppinline{llvm::LoopAnalysis},\\
			performed on a given function.
\item[\texttt{llvm::Loop}]
			Represents a single loop in a function.\\
			Contained inside a \texttt{llvm::LoopInfo}.
\end{description}

\vfill
Using \cppinline{llvm::LoopInfo} it is possible to:

\begin{itemize}
\item navigate through top-level loops:
\begin{itemize}
\item \cppinline{llvm::LoopInfo::begin()}
\item \cppinline{llvm::LoopInfo::end()}
\end{itemize}
\item get the loop for a given basic block:
\begin{itemize}
\item \cppinline{llvm::LoopInfo::operator[](llvm::BasicBlock *)}
\end{itemize}
\end{itemize}
\end{frame}


\begin{frame}{Loop Information}{Nesting Tree}
Loops are represented as a \alert{tree}:\\
\begin{columns}
\column{.6\textwidth}
\begin{block}{Source\vphantom{Loop Nest}}
\cinput[\tt\fontsize{7.5pt}{7.5pt}\selectfont]{snippet/loop-nest.c}
\end{block}
\column{.3\textwidth}
\begin{block}{Loop Hierarchy}
\centering
\input{img/loop-nest.tex}
\end{block}
\end{columns}
\vfill
\begin{description}[children loops]
\item[children loops] \cppinline{llvm::Loop::begin()},
      \cppinline{end()}
\item[parent loop] \cppinline{llvm::Loop::getParentLoop()}
\end{description}
\end{frame}


\begin{frame}{Loop Information}{Query Loops}
Accessors for important nodes:

\begin{description}
\item[pre-header] \cppinline{llvm::Loop::getLoopPreheader()}
\item[header] \cppinline{llvm::Loop::getHeader()}
\item[latch] \cppinline{llvm::Loop::getLoopLatch()}
\item[exiting] \cppinline{llvm::Loop::getExitingBlock()}, \\
               \cppinline{llvm::Loop::getExitingBlocks(...)}
\item[exit] \cppinline{llvm::Loop::getExitBlock()} \\
            \cppinline{llvm::Loop::getExitBlocks(...)}
\end{description}

\vfill
The list of all BBs in the loop is accessible via:

\begin{description}
\item[iterators] \cppinline{llvm::Loop::block\_begin()}, \\
                 \cppinline{llvm::Loop::block\_end()}
\item[vector]
      \texttt{\addfontfeature{LetterSpace=-5}std::vector<BasicBlock *> \&Loop::getBlocks()}
\end{description}
\end{frame}


\subsection{Scalar Evolution}


\begin{frame}{Scalar Evolution}
The \alert{SC}alar \alert{EV}olution pass analyzes scalar
expressions inside loops.

\begin{itemize}
\item all expressions are categorized and represented uniformly
\item is capable of handling \alert{general induction variables}
\item also useful outside of loops
\item \texttt{opt} flags: \texttt{-analyze -scalar-evolution}
\end{itemize}

\vfill
\begin{columns}[onlytextwidth]
\column{.67\textwidth}
\begin{block}{Example}
\llvminput[\tt\scriptsize]{snippet/basic-scev.ll}
\end{block}

\column{.30\textwidth}
SCEV for \llvminline{\%i.0}:

\begin{itemize}
\item initial value $0$
\item incremented \\by $1$ at each iteration
\item final value $10$
\end{itemize}
\end{columns}
\end{frame}


\begin{frame}{Scalar Evolution}{Example}
\begin{columns}
\column{.55\textwidth}
\begin{block}{Source}
\cinput[\tt\small]{snippet/nested-scev.c}
\end{block}

\column{.35\textwidth}
SCEV \llvminline{\{A,B,C\}<\%D>}:

\begin{itemize}
\item \llvminline{A} starting value
\item \llvminline{B} operator
\item \llvminline{C} stride
\item \llvminline{D} loop head BB
\end{itemize}

\llvminline{\{0,+,1\}=0+1+1+1+}\ldots

\end{columns}

\vfill
\begin{block}{Induction Variables}
\centering
\llvminput[\tt\small]{snippet/nested-scev-induction.ll}
\end{block}
\end{frame}


\begin{frame}{Scalar Evolution}{More than Induction Variables}
The scalar evolution framework manages \alert{any scalar expression}:

\begin{block}{Pointer SCEVs in two nested loops}
\centering
\llvminput[\tt\footnotesize]{snippet/nested-scev-pointer.ll}
\end{block}
%nsw = No Signed Wrap
%nuw = No Unsigned Wrap

\vfill
SCEV is an analysis used by many common optimizations
\begin{itemize}
\small
\item induction variable substitution
\item strength reduction
\item vectorization
\item \ldots
\end{itemize}
\end{frame}


\begin{frame}{Scalar Evolution}{SCEVs Design}
SCEVs are modeled by the \cppinline{llvm::SCEV} class:

\begin{itemize}
\item a subclass for each kind of SCEV: e.g. \cppinline{llvm::SCEVAddExpr}
\item instantiation disabled
\end{itemize}

\vfill
A SCEV actually is a tree of SCEVs:

\begin{itemize}
\item \llvminline{\{(80 + \%bar),+,80\}} =
\begin{itemize}
\item			\llvminline{\{\%1,+,80\}}
\item     \llvminline{\%1 = 80 + \%bar}
\end{itemize}
\end{itemize}

Tree leaves:

\begin{description}
\item[constant] \cppinline{llvm::SCEVConstant}: e.g. \llvminline{80}
\item[unknown\footnote{Not further splittable}] \cppinline{llvm::SCEVUnknown}:
                                                 e.g. \llvminline{\%bar}
\end{description}

\vfill
SCEV tree explorable through the visitor pattern:

\begin{itemize}
\item \cppinline{llvm::SCEVVisitor}
\end{itemize}
\end{frame}


\begin{frame}{Scalar Evolution}{Analysis Interface}
\centering
The \cppinline{llvm::ScalarEvolutionAnalysis} pass computes all the
SCEVs for a given \cppinline{llvm::Function}.
\vfill
\centering
The \cppinline{llvm::ScalarEvolution} instance produced by the pass
provides the following services:
\vfill
\begin{itemize}
\item get the SCEV representing a value:
\begin{itemize}
\item      \cppinline{getSCEV(llvm::Value *)}
\end{itemize}
\vfill
\item get important SCEVs from other structures or SCEVs:
\begin{itemize}
\item      \cppinline{getBackedgeTakenCount(llvm::Loop *)}
\item      \cppinline{getPointerBase(llvm::SCEV *)}
\item      \ldots
\end{itemize}
\vfill
\item create new SCEVs explicitly:
\begin{itemize}
\item \cppinline{getConstant(llvm::ConstantInt *)}
\item \cppinline{getAddExpr(llvm::SCEV *, llvm::SCEV *)}
\item \ldots
\end{itemize}
\end{itemize}
\end{frame}


\subsection{Alias Analysis}


\begin{frame}{Alias Analysis}
Let $X$ be an instruction accessing a memory location:

\begin{itemize}
\item is there another instruction accessing the same location?
\end{itemize}

\vfill
Alias analysis tries to answer the question:

\begin{description}
\item[application] optimization of memory operations
\item[problem] often fails
\end{description}

\vfill
Interface of the system: \cppinline{llvm::AAResults}

%\begin{block}{Requiring Alias Analysis}
%\centering
%\cppinput{snippet/requiring-alias-analysis.cpp}
%\end{block}
\end{frame}


\begin{frame}{Chained Analysis}
AA is actually a chain of multiple analyses, executed in sequence:
\medskip
\begin{description}[XXXXlast.]
\item[1.] Basic Alias Analysis (\texttt{basicaa})
\item[] ...
\item[n$^{\textsf{th}}$.] Type Based Alias Analysis (\texttt{tbaa})\footnote{AKA the evil alias analysis}
\item[] ...
\item[last.] Dummy Alias Analysis (\texttt{noaa})
\end{description}
\medskip
Every analysis in the chain \emph{fills the gap} left by the previous analyses.
\end{frame}


\begin{frame}{Memory Representation}
\begin{columns}
\column{.45\textwidth}
\begin{block}{Source}
\centering
\llvminput[\tt\footnotesize]{snippet/memory-locations.ll}
\end{block}

\centering
Basic building block:\\\texttt{llvm::MemoryLocation}\\
\medskip
Encapsulates a tuple:\\(\alert{address}, \alert{size})\\
\medskip
Can be computed from a \texttt{llvm::Value}

\column{.45\textwidth}
\begin{block}{Distinct Locations}
\centering
\input{img/alias-analysis-distinct.tex}
\end{block}

\begin{block}{Overlapping Locations}
\centering
\input{img/alias-analysis-overlapping.tex}
\end{block}

\begin{block}{Same Location}
\centering
\input{img/alias-analysis-same.tex}
\end{block}

\end{columns}
\end{frame}


\begin{frame}{Alias Analysis}{Basic Interface}
Given two memory locations $X$, $Y$, the alias analyzer classifies them:\\
\vfill
\begin{itemize}
\item \texttt{\textbf{llvm::AliasResult::NoAlias}}\\
			$X$ and $Y$ \alert{are
      different} memory locations
\vfill
\item \texttt{\textbf{llvm::AliasResult::MustAlias}}\\
			$X$ and $Y$ \alert{are equal}
      -- i.e. they points to the same address
\vfill
\item \texttt{\textbf{llvm::AliasResult::PartialAlias}}\\
			$X$ and $Y$
      \alert{partially overlap} -- i.e. they points to different addresses,
      but the pointed memory areas partially overlap
\vfill
\item \texttt{\textbf{llvm::AliasResult::MayAlias}}\\
			\alert{unable to compute}
      aliasing information yet -- i.e. $X$ and $Y$ can be different locations,
      or $X$ can be a complete/partial alias of $Y$
\end{itemize}

\vfill
Queries performed using:
\begin{itemize}
\item \cppinline{llvm::AAResults::alias(X, Y)}
\end{itemize}
\end{frame}


\begin{frame}{Alias Analysis}{Basic Interface}
A different categorization involves whether an instruction $I$
\alert{reads and/or modifies} a memory location $X$:\\
\vfill
\begin{itemize}
\item \texttt{\textbf{llvm::ModRefInfo::NoModRef}}\\
			The access neither references nor modifies the value stored in $X$
\vfill
\item \texttt{\textbf{llvm::ModRefInfo::Ref}}\\
			The access may reference the value stored in $X$ 
\vfill
\item \texttt{\textbf{llvm::ModRefInfo::Mod}}\\
			The access may modify the value stored in $X$ 
\vfill
\item \texttt{\textbf{llvm::ModRefInfo::ModRef}}\\
			The access may reference and may modify the value stored in $X$
\end{itemize}

\vfill
Queries performed using:
\begin{itemize}
\item \cppinline{llvm::AAResults::getModRefInfo(I, X)}
\end{itemize}
\end{frame}


\begin{frame}{Alias Analysis}{Mid-level Interface}
\begin{center}
This interface is very low-level!\\
\smallskip
What if we wanted to compute all aliases of a single value $X$?\\
\vfill
To do this, LLVM provides the \cppinline{llvm::AliasSet} class:

\begin{columns}
\column{0.75\textwidth}
\begin{enumerate}
\item instantiate a new \cppinline{llvm::AliasSetTracker}\\
			starting from \cppinline{llvm::AAResults}\footnotemark[1]
\item it builds (one or more) \cppinline{llvm::AliasSet}
\end{enumerate}
\end{columns}

\footnotetext[1]{\cppinline{using llvm::AliasAnalysis = llvm::AAResults\;}}
\vfill
For a given location $X$, a \cppinline{llvm::AliasSet}:

\begin{columns}
\column{0.6\textwidth}
\begin{itemize}
\item contains all locations aliasing with $X$
\end{itemize}
\end{columns}
\vfill
\end{center}
\end{frame}


\begin{frame}{Alias Analysis}{Alias Set Memory Accesses}
Alias sets return memory reference and aliasing information just like
the low-level interface.
\vfill
\alert{Warning:} This information is \alert{less precise}, 
as it is derived by \alert{conservatively aggregating}
more detailed data!

\vfill
\begin{itemize}
\item \texttt{\textbf{bool llvm::AliasSet::isRef()}}\\
			memory accessed in read-mode -- e.g. a
      \llvminline{load} is inside the set
\item \texttt{\textbf{bool llvm::AliasSet::isMod()}}\\
			memory accessed in write-mode -- e.g. a
      \llvminline{store} is inside the set
\vfill
\item \texttt{\textbf{bool llvm::AliasSet::isMustAlias()}}\\
			all pointers in the set \texttt{MustAlias} with each other
\item \texttt{\textbf{bool llvm::AliasSet::isMayAlias()}}\\
			at least one pair of pointer is not a \texttt{MustAlias} pair
\vfill
\end{itemize}
\end{frame}


\begin{frame}{Alias Analysis}{Mid-level Interface}
\begin{center}
Entry point is\\
\cppinline{llvm::AliasSetTracker::getAliasSetFor(...)}\\
\medskip
Only argument is a reference to\\
\cppinline{llvm::MemoryLocation}
\vfill
Once you have the \cppinline{llvm::AliasSet} you can inspect
the list of memory locations in it with the standard C++ iterator pattern:\\
\medskip
\cppinline{size()}, \cppinline{begin()}, \cppinline{end()}
\end{center}
\vfill
\end{frame}


\subsection{Memory SSA}


\begin{frame}{Memory SSA}{Alias Analyzer High-level Interface}
The \cppinline{llvm::MemorySSAAnalysis} pass wraps alias analysis to answer
queries in the following form:

\begin{itemize}
\item let \llvminline{\%foo} be an instruction accessing memory. Which
      preceding instructions does \llvminline{\%foo} depends on?
\end{itemize}

\vfill
This is done by representing all memory accesses in a\\\alert{SSA-like form}:
\begin{itemize}
\item \llvminline{store}-like instructions become \alert{definitions} (\llvminline{MemoryDef})
\item \llvminline{load}-like instructions become \alert{uses} (\llvminline{MemoryUse})
\item \llvminline{store}s to the same location in parallel CFG branches become \alert{phi}s (\llvminline{MemoryPhi})
\end{itemize}
\end{frame}


\begin{frame}{Memory Dependence Analysis}{APIs}
MemorySSA ``instructions'' are owned by \cppinline{llvm::MemorySSA} objects.\\
\bigskip
They are \alert{overlaid} on top of the normal CFG. \\
\smallskip
\begin{itemize}
\item \cppinline{AccessList *getBlockAccesses(BasicBlock *)}
\item \cppinline{DefsList *getBlockDefs(BasicBlock *)}
\end{itemize}
\bigskip
This basic interface is very \alert{hard to use}:\\
\smallskip
\begin{itemize}
\item \cppinline{llvm::MemorySSAWalker} provides support for the most common query
\item \cppinline{MemoryAccess *getClobberingMemoryAccess(...)}
\begin{itemize}
\item Returns the nearest dominating memory access that clobbers the
same memory location given.
\end{itemize}
\end{itemize}
\end{frame}

